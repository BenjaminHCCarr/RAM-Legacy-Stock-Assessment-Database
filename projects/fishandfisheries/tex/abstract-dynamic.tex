%\newpage
\section*{Abstract}

%Data used to assess the status of individual fish stocks range from
%very little information on many of the world's artisanal fisheries, to
%commercial landings, research surveys, and sophisticated population
%dynamics models that integrate many sources of information.  Previous
%evaluations of the state of global fisheries have used catch data,
%which may be poor proxies for fish stock abundances. A global
%compilation of stock assessment data in the mid-1990s enabled
%substantial syntheses of stock status; however its focus was on
%stock-recruitment relationships and it is now 15 years out of date. 

To facilitate global analyses of population dynamics and the status of
fished species, we have assembled a new database, the RAM Legacy
Database, of the most intensively studied commercially exploited
marine fish stocks. Results collated from assessment models include:
time series of total biomass, spawner biomass, recruits, fishing
mortality, and catch; reference points; and ancillary information on
the life history, management, and assessment methods for each stock.
Here, we present the first overview of this database and use it to
evaluate the knowledge-base for assessed marine species.  Assessments
were assembled for 324 stocks
(288 fish species representing
45 families, and 36
invertebrate species representing 12
families), including 8 of the world's 10 largest fisheries.
Assessments were obtained from 18 national and international
management institutions, with most relating to stocks in North
America, Europe, Australia, New Zealand and the high seas. Stocks
present in the database come from 31 Large
Marine Ecosystems and cover the Atlantic, Pacific, Indian,
Mediterranean, Arctic and Antarctic Ocean. Reference points were
available or could be calculated for about
74\% of these stocks. The available data
provide new insight into the status of exploited populations,
57\% of stocks with reference points
were estimated to be below $B_{msy}$, and
29\% had exploitation levels
estimated to be above $U_{msy}$.  Assessed marine fish stocks comprise
a relatively small proportion of harvested taxa (24\%), and an even
smaller proportion of marine fish biodiversity (1\%). We hope that
access to the database will facilitate new research in population and
fishery dynamics and life histories and encourage further data
contributions from stock assessment scientists.

%extractions from the
%database provide new insight into the status of exploited
%populations


%Globally, stock assessments were found
%for 324 stocks (288 species
%of fishes representing 45 families and
%36 species of invertebrates representing
%12 families), from 19
%national and international
%management institutions.

\noindent \textbf{Keywords}: marine fisheries, meta-analysis, population dynamics models, relational database, stock assessment, synthesis.

%with XX\% coming from north temperate regions (North
%Atlantic, North Pacific)
%\noindent Keywords: marine fisheries, meta-analysis, population dynamics models, relational database, stock assessment, synthesis.
%\newpage

%  Geographic differences in assessment
%methods show that Statistical Catch at Age (SCA) models are widely
%used by the west coast of the U.S. (XX percent of assessments),
%regional fishery management organizations in the Pacific (XX percent
%of assessments), and New Zealand (XX percent of assessments); the east
%coast of the U.S. is transitioning from Virtual Population Analysis
%(VPA) to SCA (XX percent of assessments conducted since 2000 have used
%SCA); while VPA is still the dominant assessment
%technique in western Europe (XX percent of assessments).
