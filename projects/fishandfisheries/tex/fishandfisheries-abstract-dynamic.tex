%\newpage
\section*{Abstract}

Data used to assess the status of individual fish stocks varies from
very little information on many of the world's artisanal fisheries, to
commercial landings, research surveys, and sophisticated population
dynamics models that integrate many sources of information.  Previous
evaluations of the state of global fisheries have used catch data,
which may be poor proxies for fish stock abundances. A global
compilation of stock assessment data in the mid-1990s enabled
substantial syntheses of stock status; however its focus was on
stock-recruitment relationships and it is now 15 years out of date. To
facilitate contemporary syntheses, we have assembled a new database,
the RAM Legacy Database, of the most intensively studied commercially
exploited marine fish stocks, including time series of total biomass,
spawner biomass, recruits, fishing mortality, and catch; reference
points; and ancillary information on the life history, management, and
assessment methods for each stock.  Here, we present the first
overview of this database and use it to evaluate the knowledge-base
for assessed marine species.  Assessments were assembled for
REF:SQL:TOTNUMASSESSMENT stocks (REF:SQL:TOTNUMASSESSFISH fish species
representing REF:SQL:TOTNUMFISHFAMILIES families, and
REF:SQL:TOTNUMASSESSINVERT invertebrate species representing
REF:SQL:TOTNUMINVERTFAMILIES families), including 8 of the world's 10
largest fisheries. Assessments were obtained from 18 national and
international management institutions, with most coming from North
America, Europe, Australia, New Zealand and the high seas. Reference
points were available or could be calculated for about
REF:SQL:PERCENTASSESSFRIEDEGG\% of these stocks. Overall, 58\% of
stocks with reference points were estimated to be below $B_{msy}$, and
30\% had exploitation levels estimated to be above $U_{msy}$.
Assessed marine fish stocks comprise a relatively small proportion of
harvested taxa (24\%), and an even smaller proportion of marine fish
biodiversity (1\%).


%Globally, stock assessments were found
%for REF:SQL:TOTNUMASSESSMENT stocks (REF:SQL:TOTNUMASSESSFISH species
%of fishes representing REF:SQL:TOTNUMFISHFAMILIES families and
%REF:SQL:TOTNUMASSESSINVERT species of invertebrates representing
%REF:SQL:TOTNUMINVERTFAMILIES families), from REF:SQL:TOTNUMMGMT
%national and international
%management institutions.

\noindent \textbf{Keywords}: marine fisheries, meta-analysis, population dynamics models, relational database, stock assessment, synthesis.

%with XX\% coming from north temperate regions (North
%Atlantic, North Pacific)
%\noindent Keywords: marine fisheries, meta-analysis, population dynamics models, relational database, stock assessment, synthesis.
%\newpage

%  Geographic differences in assessment
%methods show that Statistical Catch at Age (SCA) models are widely
%used by the west coast of the U.S. (XX percent of assessments),
%regional fishery management organizations in the Pacific (XX percent
%of assessments), and New Zealand (XX percent of assessments); the east
%coast of the U.S. is transitioning from Virtual Population Analysis
%(VPA) to SCA (XX percent of assessments conducted since 2000 have used
%SCA); while VPA is still the dominant assessment
%technique in western Europe (XX percent of assessments).
