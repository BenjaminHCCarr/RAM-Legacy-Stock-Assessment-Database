\section*{Results}
\subsection*{Summary}
\noindent
Total number of proper stocks assessments: REF:SQL:TOTNUMASSESSMENT, from REF:SQL:TOTNUMASSESSFISH marine fish populations and REF:SQL:TOTNUMASSESSINVERT
invertebrate populations.

\subsection*{Taxonomy}
\noindent

Number of species in FishBase: REF:SQL:NUMSPPFISHBASE\\
Number of species in SAUP: REF:SQL:SAUPNUMSPP\\
Number of species in RAM Legacy: REF:SQL:TOTNUMASSESSBYSPECIES (from REF:SQL:TOTNUMASSESSBYFAMILY families and REF:SQL:TOTNUMASSESSBYORDERS orders) \\
Top 5 taxonomic orders: REF:SQL:NUMASSESSORDERS:1, REF:SQL:NUMASSESSORDERS:2, REF:SQL:NUMASSESSORDERS:3, REF:SQL:NUMASSESSORDERS:4, REF:SQL:NUMASSESSORDERS:5 \\

\subsection*{Timespan}
\noindent
Number of assessments with catch timeseries: REF:SQL:NUMCATCHSERIES.\\
Number of assessments with recruitment timeseries: REF:SQL:NUMRSERIES.\\
Number of assessments with spawning stock biomass timeseries: REF:SQL:NUMSSBSERIES.\\

Together these comprise time series of
catch/landings for REF:SQL:NUMCATCHSERIES stocks (REF:SQL:PERCENTCATCHSERIES\%),
SSB estimates for REF:SQL:NUMSSBSERIES stocks (REF:SQL:PERCENTSSBSERIES\%), and recruitment estimates for
REF:SQL:NUMRSERIES stocks (REF:SQL:PERCENTRSERIES\%).

The median lengths of catch/landings, SSB, and recruitment timeseries
were REF:SQL:MEDCATCHLEN, REF:SQL:MEDSSBLEN, and REF:SQL:MEDRLEN
years, respectively (Figure~\ref{fig:orca}).  The time period covered by 90\% of assessments
is: catch/landings (REF:SQL:CATCHTIMESPAN90), SSB
(REF:SQL:SSBTIMESPAN90), recruitment (REF:SQL:RTIMESPAN90), while that
covered by 50\% of assessments is: catch/landings
(REF:SQL:CATCHTIMESPAN50), SSB (REF:SQL:SSBTIMESPAN50), recruitment
(REF:SQL:RTIMESPAN50)
 
\subsection*{Assessment methodologies and reference points}
\noindent
The three most common assessment methods were
REF:SQL:NUMASSESSMETHOD:1, REF:SQL:NUMASSESSMETHOD:2 and
REF:SQL:NUMASSESSMETHOD:3. Regionally, Virtual Population Analysis
(VPA) is still the most common assessment model for European stocks
(REF:SQL:PERCENTVPAICES\% of REF:SQL:TOTNUMASSESSICES assessments),
Canada (REF:SQL:PERCENTVPADFO\% of REF:SQL:TOTNUMASSESSDFO
assessments) and Argentina (REF:SQL:PERCENTVPACFP\% of
REF:SQL:TOTNUMASSESSCFP assessments), whereas statistical catch-at-age
and -length models are more common for the United States
(REF:SQL:PERCENTSCALNMFS\% of REF:SQL:TOTNUMASSESSNMFS assessments),
Australia (REF:SQL:PERCENTSCALAFMA\% of REF:SQL:TOTNUMASSESSAFMA
assessments) and New Zealand (REF:SQL:PERCENTSCALMFISH\% of
REF:SQL:TOTNUMASSESSMFish assessments).

Biomass- or exploitation-based reference points were available for
REF:SQL:NUMASSESSBIOREF (REF:SQL:PERCENTASSESSBIOREF\%) and
REF:SQL:NUMASSESSEXPLOITREF (REF:SQL:PERCENTASSESSEXPLOITREF\%)
assessments, respectively.

\subsection*{Stock status}
\noindent
Of the
REF:SQL:NUMASSESSFRIEDEGG stocks presented in
Figure~\ref{fig:friedegg}, REF:SQL:NUMASSESSBIOASSESSREF and
REF:SQL:NUMASSESSBIOSCHAEFERREF of the biomass reference points and
REF:SQL:NUMASSESSEXPLOITASSESSREF and
REF:SQL:NUMASSESSEXPLOITSCHAEFERREF of the exploitation reference
points come from assessments and from surplus production model fits,
respectively.

To identify potential biases arising from using BRPs
derived from surplus production models we computed a contingency table
of status classification for stocks that have both assessment- and
Schaefer-derived BRPs (Table S2). Surplus production models correctly
classified ratios of current biomass to BRPs in
REF:SQL:PERCENTCORRECTBCLASS\% of cases (for REF:SQL:NUMCORRECTBCLASS
of REF:SQL:NUMBCLASS assessments) and REF:SQL:PERCENTCORRECTUCLASS\%
of cases for exploitation BRPs (for REF:SQL:NUMCORRECTUCLASS of
REF:SQL:NUMUCLASS assessments).

Overall, REF:SQL:PERCENTASSESSMENTSBELOWBMSY\% of stocks are estimated
to be below their biomass-related MSY BRP, that is $B_{curr}<B_{msy}$,
and REF:SQL:PERCENTASSESSMENTSABOVEFMSY\% are estimated to be above
their exploitation-related MSY BRP, $U_{curr}>U_{msy}$
(n=REF:SQL:NUMASSESSFRIEDEGG stocks total; Figure~\ref{fig:friedegg}).
Of the stocks for which biomass is currently estimated to be below
$B_{msy}$, REF:SQL:PERCENTASSESSBELOWBMSYANDBELOWFMSY\% have had their
exploitation rate reduced below $U_{msy}$, suggesting potential for
recovery (Figure~\ref{fig:friedegg}). The remaining
REF:SQL:PERCENTASSESSBELOWBMSYANDABOVEFMSY\% of these stocks however,
still have excessive exploitation rates
(Figure~\ref{fig:friedegg}). On a positive note,
REF:SQL:PERCENTASSESSMENTSABOVEBMSY\% of all stocks are estimated to
be above $B_{msy}$, and REF:SQL:PERCENTASSESSABOVEBMSYANDBELOWFMSY\%
of the stocks above $B_{msy}$ also have $U_{current}$ below $U_{msy}$.


\subsection*{Global fisheries}

\subsection*{Management bodies and geography}
\noindent
Number of assessments from NMFS: REF:SQL:TOTNUMASSESSNMFS (REF:SQL:NUMASSESSFRIEDTOTALNMFS with reference points, REF:SQL:NUMASSESSFRIEDBELOWBMSYNMFS (REF:SQL:PERCENTASSESSFRIEDBELOWBMSYNMFS \%) are below $B_{msy}$, REF:SQL:NUMASSESSFRIEDBELOWFMSYNMFS (REF:SQL:PERCENTASSESSFRIEDBELOWFMSYNMFS \%) are below $U_{msy}$, ) \\
Number of assessments from ICES: REF:SQL:TOTNUMASSESSICES (REF:SQL:NUMASSESSFRIEDTOTALICES with reference points, REF:SQL:NUMASSESSFRIEDBELOWBMSYICES (REF:SQL:PERCENTASSESSFRIEDBELOWBMSYICES \%) are below $B_{msy}$, REF:SQL:NUMASSESSFRIEDBELOWFMSYICES (REF:SQL:PERCENTASSESSFRIEDBELOWFMSYICES \%) are below $U_{msy}$, ) \\
Number of assessments from MFish: REF:SQL:TOTNUMASSESSMFish (REF:SQL:NUMASSESSFRIEDTOTALMFish with reference points, REF:SQL:NUMASSESSFRIEDBELOWBMSYMFish (REF:SQL:PERCENTASSESSFRIEDBELOWBMSYMFish \%) are below $B_{msy}$, REF:SQL:NUMASSESSFRIEDBELOWFMSYMFish (REF:SQL:PERCENTASSESSFRIEDBELOWFMSYMFish \%) are below $U_{msy}$, ) \\
Number of assessments from DFO: REF:SQL:TOTNUMASSESSDFO (REF:SQL:NUMASSESSFRIEDTOTALDFO with reference points, REF:SQL:NUMASSESSFRIEDBELOWBMSYDFO (REF:SQL:PERCENTASSESSFRIEDBELOWBMSYDFO \%) are below $B_{msy}$, REF:SQL:NUMASSESSFRIEDBELOWFMSYDFO (REF:SQL:PERCENTASSESSFRIEDBELOWFMSYDFO \%) are below $U_{msy}$, ) \\
Number of assessments from AFMA: REF:SQL:TOTNUMASSESSAFMA (REF:SQL:NUMASSESSFRIEDTOTALAFMA with reference points, REF:SQL:NUMASSESSFRIEDBELOWBMSYAFMA (REF:SQL:PERCENTASSESSFRIEDBELOWBMSYAFMA \%) are below $B_{msy}$, REF:SQL:NUMASSESSFRIEDBELOWFMSYAFMA (REF:SQL:PERCENTASSESSFRIEDBELOWFMSYAFMA \%) are below $U_{msy}$, ) \\
Number of assessments from DETMCM: REF:SQL:TOTNUMASSESSDETMCM (REF:SQL:NUMASSESSFRIEDTOTALDETMCM with reference points, REF:SQL:NUMASSESSFRIEDBELOWBMSYDETMCM (REF:SQL:PERCENTASSESSFRIEDBELOWBMSYDETMCM \%) are below $B_{msy}$, REF:SQL:NUMASSESSFRIEDBELOWFMSYDETMCM (REF:SQL:PERCENTASSESSFRIEDBELOWFMSYDETMCM \%) are below $U_{msy}$, ) \\

The status of exploited marine stocks, as estimated from biomass- and
exploitaion-BRPs, varied widely depending on the management body
(Figure~\ref{fig:friedegg}). Most European stocks (managed by
ICES) have biomasses less than $B_{msy}$
(REF:SQL:PERCENTASSESSFRIEDBELOWBMSYICES\%), and over half of these
stocks (REF:SQL:PERCENTASSESSFRIEDBELBMSYANDABOVEFMSYICES\%) still
have exploitation rates exceeding $U_{msy}$. Canadian stocks (managed
by DFO) also had low biomass (REF:SQL:PERCENTASSESSFRIEDBELOWBMSYDFO\%
$< B_{msy}$), but all but one of these has had its exploitation rate
reduced below $U_{msy}$. In contrast, about half
(REF:SQL:PERCENTASSESSFRIEDABOVEBMSYNMFS\%) of U.S. stocks (managed by
NMFS) are estimated to still be above $B_{msy}$, and of the
REF:SQL:NUMASSESSFRIEDBELOWBMSYNMFS stocks that are below $B_{msy}$
REF:SQL:PERCENTASSESSFRIEDBELBMSYANDBELOWFMSYNMFS\% have exploitation
rates below $U_{msy}$ (Figure~\ref{fig:friedegg}). In the New
Zealand and Australian waters, stocks managed by MFish and AFMA are
above $B_{msy}$ in REF:SQL:PERCENTASSESSFRIEDABOVEBMSYMFish\% and
REF:SQL:PERCENTASSESSFRIEDABOVEBMSYAFMA\% of cases, respectively. For
the stocks grouped as ``Atlantic'' in Figure~\ref{fig:friedegg} we
found that REF:SQL:NUMASSESSFRIEDBELOWBMSYICCAT of the
REF:SQL:NUMASSESSFRIEDTOTALICCAT ICCAT stocks and
REF:SQL:NUMASSESSFRIEDBELOWBMSYICCAT of the
REF:SQL:NUMASSESSFRIEDTOTALICCAT of NAFO stocks were below $B_{msy}$ .

%Number of assessments from ICES: REF:SQL:TOTNUMASSESSICES.\\
%Number of assessments from MFish: REF:SQL:TOTNUMASSESSMFish.\\
%Number of assessments from DFO: REF:SQL:TOTNUMASSESSDFO.\\
%Number of assessments from AFMA: REF:SQL:TOTNUMASSESSAFMA.\\
%Number of assessments from DETMCM: REF:SQL:TOTNUMASSESSDETMCM.\\


Assessments were available for REF:SQL:TOTNUMLMESPRIMARY LMEs, with the greatest number of
assessed stocks coming from REF:SQL:NUMASSESSLME:1,
REF:SQL:NUMASSESSLME:2, REF:SQL:NUMASSESSLME:3,
REF:SQL:NUMASSESSLME:4, REF:SQL:NUMASSESSLME:5, REF:SQL:NUMASSESSLME:6
and REF:SQL:NUMASSESSLME:7.

The proportion of stocks below $B_{msy}$ and below $U_{mys}$ varies considerably by management body. 

ICES has REF:SQL:NUMASSESSFRIEDTOTALICES assessments in Table 4,
REF:SQL:NUMASSESSFRIEDBELOWBMSYICES
(REF:SQL:PERCENTASSESSFRIEDBELOWBMSYICES\%) of which are below
$B_{msy}$ and REF:SQL:NUMASSESSFRIEDBELOWFMSYICES are below
$U_{msy}$.
