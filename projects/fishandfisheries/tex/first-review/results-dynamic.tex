\section*{Results}
\subsection*{Summary}
\noindent
Total number of proper stocks assessments: 327, from 291 marine fish populations and 36
invertebrate populations.

\subsection*{Taxonomy}
\noindent

Number of species in FishBase: 12339\\
Number of species in SAUP: 925\\
Number of species in RAM Legacy: 160 (from 57 families and 20 orders) \\
Top 5 taxonomic orders: Gadiformes (n=69), Perciformes (n=62), Pleuronectiformes (n=53), Scorpaeniformes (n=40), Clupeiformes (n=37) \\

\subsection*{Timespan}
\noindent
Number of assessments with catch timeseries: 310.\\
Number of assessments with recruitment timeseries: 271.\\
Number of assessments with spawning stock biomass timeseries: 277.\\

Together these comprise time series of
catch/landings for 310 stocks (95\%),
SSB estimates for 277 stocks (85\%), and recruitment estimates for
271 stocks (83\%).

The median lengths of catch/landings, SSB, and recruitment timeseries
were 38, 34, and 33
years, respectively (Figure~\ref{fig:orca}).  The time period covered by 90\% of assessments
is: catch/landings (1967-2007), SSB
(1972-2007), recruitment (1971-2006), while that
covered by 50\% of assessments is: catch/landings
(1983-2004), SSB (1985-2005), recruitment
(1984-2003)
 
\subsection*{Assessment methodologies and reference points}
\noindent
The three most common assessment methods were
Statistical catch-at-age/length models (n=164), Virtual Population Analyses (n=92) and
Biomass dynamics model (n=45). Regionally, Virtual Population Analysis
(VPA) is still the most common assessment model for European stocks
(71\% of 63 assessments),
Canada (58\% of 24
assessments) and Argentina (83\% of
6 assessments), whereas statistical catch-at-age
and -length models are more common for the United States
(66\% of 139 assessments),
Australia (81\% of 16
assessments) and New Zealand (76\% of
29 assessments).

Biomass- or exploitation-based reference points were available for
257 (80\%) and
221 (68\%)
assessments, respectively.

\subsection*{Stock status}
\noindent
Of the
213 stocks presented in
Figure~\ref{fig:friedegg}, 107 and
106 of the biomass reference points and
79 and
134 of the exploitation reference
points come from assessments and from surplus production model fits,
respectively.

To identify potential biases arising from using BRPs
derived from surplus production models we computed a contingency table
of status classification for stocks that have both assessment- and
Schaefer-derived BRPs (Table S2). Surplus production models correctly
classified ratios of current biomass to BRPs in
76\% of cases (for 59
of 78 assessments) and 65\%
of cases for exploitation BRPs (for 30 of
46 assessments).

Overall, 59\% of stocks are estimated
to be below their biomass-related MSY BRP, that is $B_{curr}<B_{msy}$,
and 30\% are estimated to be above
their exploitation-related MSY BRP, $U_{curr}>U_{msy}$
(n=213 stocks total; Figure~\ref{fig:friedegg}).
Of the stocks for which biomass is currently estimated to be below
$B_{msy}$, 56\% have had their
exploitation rate reduced below $U_{msy}$, suggesting potential for
recovery (Figure~\ref{fig:friedegg}). The remaining
44\% of these stocks however,
still have excessive exploitation rates
(Figure~\ref{fig:friedegg}). On a positive note,
41\% of all stocks are estimated to
be above $B_{msy}$, and 91\%
of the stocks above $B_{msy}$ also have $U_{current}$ below $U_{msy}$.


\subsection*{Global fisheries}

\subsection*{Management bodies and geography}
\noindent
Number of assessments from NMFS: 139 (80 with reference points, 40 (50 \%) are below $B_{msy}$, 63 (79 \%) are below $U_{msy}$, ) \\
Number of assessments from ICES: 63 (48 with reference points, 39 (81 \%) are below $B_{msy}$, 22 (46 \%) are below $U_{msy}$, ) \\
Number of assessments from MFish: 29 (28 with reference points, 11 (39 \%) are below $B_{msy}$, 22 (79 \%) are below $U_{msy}$, ) \\
Number of assessments from DFO: 24 (15 with reference points, 13 (87 \%) are below $B_{msy}$, 15 (100 \%) are below $U_{msy}$, ) \\
Number of assessments from AFMA: 16 (10 with reference points, 7 (70 \%) are below $B_{msy}$, 6 (60 \%) are below $U_{msy}$, ) \\
Number of assessments from DETMCM: 14 (6 with reference points, 3 (50 \%) are below $B_{msy}$, 5 (83 \%) are below $U_{msy}$, ) \\

The status of exploited marine stocks, as estimated from biomass- and
exploitaion-BRPs, varied widely depending on the management body
(Figure~\ref{fig:friedegg}). Most European stocks (managed by
ICES) have biomasses less than $B_{msy}$
(81\%), and over half of these
stocks (59\%) still
have exploitation rates exceeding $U_{msy}$. Canadian stocks (managed
by DFO) also had low biomass (87\%
$< B_{msy}$), but all but one of these has had its exploitation rate
reduced below $U_{msy}$. In contrast, about half
(50\%) of U.S. stocks (managed by
NMFS) are estimated to still be above $B_{msy}$, and of the
40 stocks that are below $B_{msy}$
65\% have exploitation
rates below $U_{msy}$ (Figure~\ref{fig:friedegg}). In the New
Zealand and Australian waters, stocks managed by MFish and AFMA are
above $B_{msy}$ in 61\% and
30\% of cases, respectively. For
the stocks grouped as ``Atlantic'' in Figure~\ref{fig:friedegg} we
found that 6 of the
10 ICCAT stocks and
6 of the
10 of NAFO stocks were below $B_{msy}$ .

%Number of assessments from ICES: 63.\\
%Number of assessments from MFish: 29.\\
%Number of assessments from DFO: 24.\\
%Number of assessments from AFMA: 16.\\
%Number of assessments from DETMCM: 14.\\


Assessments were available for 32 LMEs, with the greatest number of
assessed stocks coming from Northeast U.S. Continental Shelf (n=58),
California Current (n=35), New Zealand Shelf (n=29),
Celtic-Biscay Shelf (n=26), Gulf of Alaska (n=26), East Bering Sea (n=22)
and Southeast U.S. Continental Shelf (n=20).

The proportion of stocks below $B_{msy}$ and below $U_{mys}$ varies considerably by management body. 

ICES has 48 assessments in Table 4,
39
(81\%) of which are below
$B_{msy}$ and 22 are below
$U_{msy}$.
