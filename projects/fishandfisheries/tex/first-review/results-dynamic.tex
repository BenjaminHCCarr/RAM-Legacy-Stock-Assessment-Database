\section*{Results}
\subsection*{Summary}
\noindent
Total number of proper stocks assessments: 326, from 289 marine fish populations and 37
invertebrate populations.


\subsection*{Global fisheries}

\subsection*{Management bodies and geography}
\noindent
Number of assessments from NMFS: 140.\\
Number of assessments from ICES: 63.\\
Number of assessments from MFish: 29.\\
Number of assessments from DFO: 22.\\
Number of assessments from AFMA: 16.\\
Number of assessments from DETMCM: 14.\\


Assessments were available for 32 LMEs, with the greatest number of
assessed stocks coming from Northeast U.S. Continental Shelf (n=59),
California Current (n=35), New Zealand Shelf (n=29),
Celtic-Biscay Shelf (n=26), Gulf of Alaska (n=26), East Bering Sea (n=22)
and Southeast U.S. Continental Shelf (n=20).

\subsection*{Taxonomy}
\noindent

Number of species in FishBase: 12339\\
Number of species in SAUP: 649\\
Number of species in RAM Legacy: 160 (from 57 families and 20 orders) \\
Top 5 taxonomic orders: Gadiformes (n=67), Perciformes (n=62), Pleuronectiformes (n=53), Scorpaeniformes (n=40), Clupeiformes (n=37) \\

\subsection*{Timespan}
\noindent
Number of assessments with catch timeseries: 309.\\
Number of assessments with recruitment timeseries: 272.\\
Number of assessments with spawning stock biomass timeseries: 275.\\

The median lengths of catch/landings, SSB, and recruitment timeseries
were 38, 34, and 33
years, respectively (Figure~\ref{fig:orca}).  The time period covered by 90\% of assessments
is: catch/landings (1967-2007), SSB
(1972-2007), recruitment (1971-2006), while that
covered by 50\% of assessments is: catch/landings
(1983-2004), SSB (1985-2005), recruitment
(1984-2003)
 
\subsection*{Assessment methodologies and reference points}
\noindent
The three most common assessment methods were
Statistical catch-at-age/length models (n=164), Virtual Population Analyses (n=91) and
Biomass dynamics model (n=46). Regionally, Virtual Population Analysis
(VPA) is still the most common assessment model for European stocks
(71\% of 63 assessments),
Canada (59\% of 22
assessments) and Argentina (83\% of
6 assessments), whereas statistical catch-at-age
and -length models are more common for the United States
(66\% of 140 assessments),
Australia (81\% of 16
assessments) and New Zealand (76\% of
29 assessments).

Biomass- or exploitation-based reference points were available for
257 (80\%) and
222 (69\%)
assessments, respectively.

\subsection*{Stock status}
\noindent
Of the
240 stocks presented in
Figure~\ref{fig:friedegg}, 112 and
128 of the biomass reference points and
83 and
157 of the exploitation reference
points come from assessments and from surplus production model fits,
respectively.

Overall, 57\% of stocks are estimated
to be below their biomass-related MSY BRP, that is $B_{curr}<B_{msy}$,
and 30\% are estimated to be above
their exploitation-related MSY BRP, $U_{curr}>U_{msy}$
(n=240 stocks total; Figure~\ref{fig:friedegg}).
Of the stocks for which biomass is currently estimated to be below
$B_{msy}$, 55\% have had their
exploitation rate reduced below $U_{msy}$, suggesting potential for
recovery (Figure~\ref{fig:friedegg}). The remaining
45\% of these stocks however,
still have excessive exploitation rates (Figure~\ref{fig:friedegg}).
On a positive note, 43\% of all stocks are
estimated to be above $B_{msy}$, and
91\% of the stocks above
$B_{msy}$ also have $U_{current}$ below $U_{msy}$.
