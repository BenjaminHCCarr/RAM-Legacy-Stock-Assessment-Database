\section*{Results}
\subsection*{Summary}
\noindent
Total number of proper stocks assessments: 331, from 295 marine fish populations and 36
invertebrate populations.

\subsection*{Taxonomy}
\noindent

Number of species in FishBase: 12339 (from 54 orders) \\
Number of species in SAUP: 925 (from 36 orders)\\
Number of species in RAM Legacy: 163 (from 58 families and 20 orders) \\
RAM Legacy contains 18\% of SAUP and 1\% of FishBase species\\
Top 5 taxonomic orders in RAM Legacy: Gadiformes (n=70), Perciformes (n=65), Pleuronectiformes (n=53), Scorpaeniformes (n=41), Clupeiformes (n=36) \\

\subsection*{Timespan}
\noindent
Number of assessments with catch timeseries: 313.\\
Number of assessments with recruitment timeseries: 274.\\
Number of assessments with spawning stock biomass timeseries: 280.\\

Together these comprise time series of
catch/landings for 313 stocks (95\%),
SSB estimates for 280 stocks (85\%), and recruitment estimates for
274 stocks (83\%).

The median lengths of catch/landings, SSB, and recruitment timeseries
were 39, 34, and 33
years, respectively.  The time period covered by 90\% of assessments
is: catch/landings (1966-2007), SSB
(1972-2007), recruitment (1971-2006), while that
covered by 50\% of assessments is: catch/landings
(1983-2004), SSB (1985-2005), recruitment
(1984-2003)
 
\subsection*{Assessment methodologies and reference points}
\noindent
The three most common assessment methods were
Statistical catch-at-age/length models (n=168), Virtual Population Analyses (n=92) and
Biomass dynamics model (n=46). Regionally, Virtual Population Analysis
(VPA) is still the most common assessment model for European stocks
(71\% of 63 assessments),
Canada (56\% of 26
assessments) and Argentina (83\% of
6 assessments), whereas statistical catch-at-age
and -length models are more common for the United States
(67\% of 138 assessments),
Australia (82\% of 17
assessments) and New Zealand (76\% of
29 assessments).

Biomass- or exploitation-based reference points were available for
262 (81\%) and
224 (69\%)
assessments, respectively.

\subsection*{Stock status}
\noindent

MSY-related reference points were avaialble for
110 stocks
(3 invertebrates) and estimated
for 104 additional stocks
(15 invertebrates), for a total of
214 stocks.

Of the
214 stocks presented in
the fried egg, 110 and
104 of the biomass reference points and
82 and
132 of the exploitation reference
points come from assessments and from surplus production model fits,
respectively.

To identify potential biases arising from using BRPs
derived from surplus production models we computed a contingency table
of status classification for stocks that have both assessment- and
Schaefer-derived BRPs (Table S2). Surplus production models correctly
classified ratios of current biomass to BRPs in
76\% of cases (for 58
of 76 assessments) and 64\%
of cases for exploitation BRPs (for 28 of
44 assessments).

Overall, 58\% of stocks are estimated
to be below their biomass-related MSY BRP, that is $B_{curr}<B_{msy}$,
and 30\% are estimated to be above
their exploitation-related MSY BRP, $U_{curr}>U_{msy}$
(n=214 stocks total.
Of the stocks for which biomass is currently estimated to be below
$B_{msy}$, 55\% have had their
exploitation rate reduced below $U_{msy}$, suggesting potential for
recovery. The remaining
45\% of these stocks however,
still have excessive exploitation rates. On a positive note,
42\% of all stocks are estimated to
be above $B_{msy}$, and 91\%
of the stocks above $B_{msy}$ also have $U_{current}$ below $U_{msy}$.


\subsection*{Global fisheries}

\subsection*{Management bodies and geography}
\noindent
Number of assessments from NMFS: 138 (80 with reference points, 40 (50 \%) are below $B_{msy}$, 63 (79 \%) are below $U_{msy}$, ) \\

Number of assessments from ICES: 63 (48 with reference points, 39 (81 \%) are below $B_{msy}$, 22 (46 \%) are below $U_{msy}$, ) \\

Number of assessments from ICES: 63 (23 with Blim and Flim reference points, 7 are below $B_{lim}$ and above $F_{lim}$, 1 are above $B_{lim}$ and above $F_{lim}$, 11 are above $B_{lim}$ and below $F_{lim}$ and 4 are below $B_{msy}$ and below $F_{lim}$.

Number of assessments from MFish: 29 (28 with reference points, 11 (39 \%) are below $B_{msy}$, 22 (79 \%) are below $U_{msy}$, ) \\
Number of assessments from DFO: 26 (14 with reference points, 12 (86 \%) are below $B_{msy}$, 13 (93 \%) are below $U_{msy}$, ) \\
Number of assessments from AFMA: 17 (11 with reference points, 7 (64 \%) are below $B_{msy}$, 7 (64 \%) are below $U_{msy}$, ) \\
Number of assessments from DETMCM: 14 (6 with reference points, 3 (50 \%) are below $B_{msy}$, 5 (83 \%) are below $U_{msy}$, ) \\

The status of exploited marine stocks, as estimated from biomass- and
exploitaion-BRPs, varied widely depending on the management body. Most European stocks (managed by
ICES) have biomasses less than $B_{msy}$
(81\%), and over half of these
stocks (59\%) still
have exploitation rates exceeding $U_{msy}$. Canadian stocks (managed
by DFO) also had low biomass (86\%
$< B_{msy}$), but all but one of these has had its exploitation rate
reduced below $U_{msy}$. In contrast, about half
(50\%) of U.S. stocks (managed by
NMFS) are estimated to still be above $B_{msy}$, and of the
40 stocks that are below $B_{msy}$
65\% have exploitation
rates below $U_{msy}$. In the New
Zealand and Australian waters, stocks managed by MFish and AFMA are
above $B_{msy}$ in 61\% and
36\% of cases, respectively. For
the stocks grouped as ``Atlantic'' in the fried eggs we
found that 6 of the
10 ICCAT stocks and
6 of the
10 of NAFO stocks were below $B_{msy}$ .

%Number of assessments from ICES: 63.\\
%Number of assessments from MFish: 29.\\
%Number of assessments from DFO: 26.\\
%Number of assessments from AFMA: 17.\\
%Number of assessments from DETMCM: 14.\\


Assessments were available for 27 LMEs, with the greatest number of
assessed stocks coming from Northeast U.S. Continental Shelf (n=59),
California Current (n=35), New Zealand Shelf (n=29),
Gulf of Alaska (n=27), Celtic-Biscay Shelf (n=26), East Bering Sea (n=21)
and Southeast U.S. Continental Shelf (n=20).

The proportion of stocks below $B_{msy}$ and below $U_{mys}$ varies considerably by management body. 

ICES has 48 assessments in Table 4,
39
(81\%) of which are below
$B_{msy}$ and 22 are below
$U_{msy}$.

\subsection*{Stock status by taxonomic orders}

Of the 48 stocks for Gadiformes, 15 are below $B_{msy}$ and above $U_{msy}$, 2 are above $B_{msy}$ and above $U_{msy}$, 9 are above $B_{msy}$ and below $U_{msy}$ and 22 are below $B_{msy}$ and below $U_{msy}$.

Of the 45 stocks for Perciformes, 13 are below $B_{msy}$ and above $U_{msy}$, 1 are above $B_{msy}$ and above $U_{msy}$, 17 are above $B_{msy}$ and below $U_{msy}$ and 14 are below $B_{msy}$ and below $U_{msy}$.

Of the 38 stocks for Pleuronectiformes, 14 are below $B_{msy}$ and above $U_{msy}$, 1 are above $B_{msy}$ and above $U_{msy}$, 18 are above $B_{msy}$ and below $U_{msy}$ and 5 are below $B_{msy}$ and below $U_{msy}$.

Of the 25 stocks for Scorpaeniformes, 2 are below $B_{msy}$ and above $U_{msy}$, 1 are above $B_{msy}$ and above $U_{msy}$, 14 are above $B_{msy}$ and below $U_{msy}$ and 8 are below $B_{msy}$ and below $U_{msy}$.

Of the 23 stocks for Clupeiformes, 4 are below $B_{msy}$ and above $U_{msy}$, 2 are above $B_{msy}$ and above $U_{msy}$, 7 are above $B_{msy}$ and below $U_{msy}$ and 10 are below $B_{msy}$ and below $U_{msy}$.

Of the 12 stocks for Decapoda, 5 are below $B_{msy}$ and above $U_{msy}$, 1 are above $B_{msy}$ and above $U_{msy}$, 2 are above $B_{msy}$ and below $U_{msy}$ and 4 are below $B_{msy}$ and below $U_{msy}$.


\subsection*{Stock status by Mean Trophic Level}
Of the 26 stocks of MTL between 2 and 3 , 10 are below $B_{msy}$ and above $U_{msy}$, 1 are above $B_{msy}$ and above $U_{msy}$, 7 are above $B_{msy}$ and below $U_{msy}$ and 8 are below $B_{msy}$ and below $U_{msy}$.

Of the 94 stocks of MTL between 3 and 4 , 19 are below $B_{msy}$ and above $U_{msy}$, 3 are above $B_{msy}$ and above $U_{msy}$, 38 are above $B_{msy}$ and below $U_{msy}$ and 34 are below $B_{msy}$ and below $U_{msy}$.

Of the 89 stocks of MTL above 4 , 25 are below $B_{msy}$ and above $U_{msy}$, 4 are above $B_{msy}$ and above $U_{msy}$, 34 are above $B_{msy}$ and below $U_{msy}$ and 26 are below $B_{msy}$ and below $U_{msy}$.

\subsection*{Stock status by Functional Grouping}
Of the 146 demersal stocks, 39 are below $B_{msy}$ and above $U_{msy}$, 4 are above $B_{msy}$ and above $U_{msy}$, 58 are above $B_{msy}$ and below $U_{msy}$ and 45 are below $B_{msy}$ and below $U_{msy}$.

Of the 49 pelagic stocks, 10 are below $B_{msy}$ and above $U_{msy}$, 3 are above $B_{msy}$ and above $U_{msy}$, 18 are above $B_{msy}$ and below $U_{msy}$ and 18 are below $B_{msy}$ and below $U_{msy}$.

Of the 18 invertebrates stocks, 7 are below $B_{msy}$ and above $U_{msy}$, 1 are above $B_{msy}$ and above $U_{msy}$, 4 are above $B_{msy}$ and below $U_{msy}$ and 6 are below $B_{msy}$ and below $U_{msy}$.
