\subsection*{Geographic coverage}
In total, REF:SQL:TOTNUMASSESSMENT recent stock assessments for
REF:SQL:TOTNUMASSESSFISH marine fish and REF:SQL:TOTNUMASSESSINVERT
invertebrate populations are included in the RAM Legacy database
(Version 1.0, 2010). These include all stocks assessed by fisheries
agencies in European Countries (International Council for the
Exploration of the Sea (ICES), n=REF:SQL:TOTNUMASSESSICES), the United
States (National Marine Fisheries Service (NMFS),
n=REF:SQL:TOTNUMASSESSNMFS), Canada (Department of Fisheries and
Oceans (DFO), n=REF:SQL:TOTNUMASSESSDFO), New Zealand (Ministry of
Fisheries, n=REF:SQL:TOTNUMASSESSMFish), Australia (Australian
Fisheries Management Authority (AFMA) n=REF:SQL:TOTNUMASSESSAFMA),
South Africa (Department of Environment and Tourism, Marine and
Coastal Management (DETMCM), n=REF:SQL:TOTNUMASSESSDETMCM) and
Argentina (Consejo Federal Pesquero, n=REF:SQL:TOTNUMASSESSCFP).  Also
included are assessments conducted by Regional Fisheries Management
Organizations (RFMOs) in the Northwest Atlantic (Northwest Atlantic
Fisheries Organization (NAFO), n= REF:SQL:TOTNUMASSESSNAFO), Atlantic
(International Commission for the Conservation of Atlantic Tunas
(ICCAT), n=REF:SQL:TOTNUMASSESSICCAT), Pacific (Western and Central
Pacific Fisheries Commission, n=REF:SQL:TOTNUMASSESSWCPFC and South
Pacific Regional Fisheries Management Organization,
n=REF:SQL:TOTNUMASSESSSPRFMO and Inter-American Tropical Tuna
Commission, n=REF:SQL:TOTNUMASSESSIATTC and International Pacific
Halibut Commission, n=REF:SQL:TOTNUMASSESSIPHC) and Indian Ocean
(Indian Ocean Tuna Commission, n=REF:SQL:TOTNUMASSESSIOTC). \\
The three Large Marine Ecosystems (LMEs) with the most number of assessed populations entered are the: REF:SQL:NUMASSESSLME:1, REF:SQL:NUMASSESSLME:2 and REF:SQL:NUMASSESSLME:3 (Figure~\ref{fig:lmes}). \\
\subsection*{Taxonomic coverage}
The taxonomic coverage of the database includes REF:SQL:TOTNUMASSESSBYSPECIES species from REF:SQL:TOTNUMASSESSBYFAMILY families (Figure~\ref{fig:taxo:srdb}). This comprises a relatively small proportion of caught taxa and a smaller proportion of marine fish biodiversity (Figure~\ref{fig:taxo:fourpanel}).\\

Four taxonomic orders (Gadiformes, Pleuronectiformes Perciformes and Scorpaeniformes) account for 67\% of available stock assessments.

%Of the species caught in reported fisheries, the taxonomic coverage of the RAM Legacy database only covers XXpercent of taxonomic orders. 

\subsection*{Temporal coverage}
The number and median lengths of timeseries stored in the RAM Legacy database are catch/landings: n=REF:SQL:NUMCATCHSERIES, length=REF:SQL:MEDCATCHLEN years, SSB: n=REF:SQL:NUMSSBSERIES, length=REF:SQL:MEDSSBLEN years, and recruitment: n=REF:SQL:NUMRSERIES, length=REF:SQL:MEDRLEN years (Figure~\ref{fig:orca}).
\subsection*{Assessment methodology}
Of the REF:SQL:TOTNUMASSESSMENT assessments in the database, REF:SQL:TOTNUMWITHMODEL use a proper population dynamics model while the remaining REF:SQL:TOTNUMWITHOUTMODEL are based on scientific survey information. The most common assessment methods were REF:SQL:NUMASSESSMETHOD:1, REF:SQL:NUMASSESSMETHOD:2 and REF:SQL:NUMASSESSMETHOD:3.

\subsection*{Reference points and life history information}
The total number and percentage of assessments that reported biomass- or exploitation- based reference points was REF:SQL:NUMASSESSBIOREF (REF:SQL:PERCENTASSESSBIOREF\%) and REF:SQL:NUMASSESSEXPLOITREF (REF:SQL:PERCENTASSESSEXPLOITREF\%), respectively. While the total number and percentage of assessments that reported any life-history information (growth, maturity, fecundity) was REF:SQL:NUMASSESSLIFE.

%  (REF:SQL:PERCENTASSESSLIFE\%), respectively


%-distinction in assessments: do they exist and we have just not entered them? or do they not exist?

% num of marine fish populations for there are stock assessments:

% -taxonomic coverage (family, species), trophic level (and/or pelagic, demersal.....), habitat, diversity, ....   

% number of the top 10 (or top 100) fisheries for which there are stock assessments (number of these that are for marine fishes vs. marine invertebrates) 

% Commercial?: -those that are commercially valuable, or some incidentally caught species of conservation concern (e.g. sharks, ...?)

% -types of assessment methodology: what percent are VPA, age-stuctured, production models etc. and break down by country 

% -which ones have reference points? 

% -difference between the maximum and minimum of each stock (>50 percent i.e. is this why they are assessed?; but how much would they vary just with natural variability?) 
