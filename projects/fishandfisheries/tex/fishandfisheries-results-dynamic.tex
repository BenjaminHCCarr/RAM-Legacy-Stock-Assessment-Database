\newpage
\section*{Results}
%\subsection*{Scope of the RAM Legacy database}
\subsection*{The knowledge-base for commercially-exploited marine stocks}
In total, REF:SQL:TOTNUMASSESSMENT recent stock assessments for
REF:SQL:TOTNUMASSESSFISH marine fish and REF:SQL:TOTNUMASSESSINVERT
invertebrate populations are included in the RAM Legacy database
(Version 1.0, 2010; Table S1). Together these comprise time series of
catch/landings for REF:SQL:NUMCATCHSERIES stocks (REF:SQL:PERCENTCATCHSERIES\%),
SSB estimates for REF:SQL:NUMSSBSERIES stocks (REF:SQL:PERCENTSSBSERIES\%), and recruitment estimates for
REF:SQL:NUMRSERIES stocks (REF:SQL:PERCENTRSERIES\%) (Table S1).

\subsubsection*{Management bodies and geography}
Stock assessments are derived from fisheries management bodies in
Europe, the United States, Canada, New Zealand, Australia, Russia,
South Africa and Argentina (Table~\ref{tab:mgmt}). Also included are
assessments conducted by eight Regional Fisheries Management
Organizations (RFMOs), in the Northwest Atlantic, Atlantic, Pacific
and Indian Ocean (Table~\ref{tab:mgmt}). Assessments from the United
States constitute by far the most stocks of any country or region
(n=REF:SQL:TOTNUMASSESSNMFS); assessments from the European Union's
management body, the International Council for the Exploration of the
Seas (ICES), constitute the the second greatest number of stocks
(n=REF:SQL:TOTNUMASSESSICES).  Whereas nations are responsible for
managing all populations within their EEZs, RFMOs typically focus on a
certain type of species (e.g.  halibut, tunas) or fisheries (e.g.
pelagic high seas) within a given area and hence assess a smaller
number of stocks.

Most assessments come from North America, Europe, Australia, New
Zealand and the high seas, while there are few from regions such as
Southeast Asia, South America, and the Indian Ocean (outside
Australian waters) (Figure~\ref{fig:lmes}). Assessments were available for REF:SQL:TOTNUMLMESPRIMARY LMEs, with the greatest number of
assessed stocks coming from REF:SQL:NUMASSESSLME:1,
REF:SQL:NUMASSESSLME:2, REF:SQL:NUMASSESSLME:3,
REF:SQL:NUMASSESSLME:4, REF:SQL:NUMASSESSLME:5, REF:SQL:NUMASSESSLME:6
and REF:SQL:NUMASSESSLME:7 (Figure~\ref{fig:lmes}).

%Northeast U.S. Continental Shelf (n=88), the California Current (n=35, the East Bering Sea (n=32), the New Zealand Shelf (n=29), the Gulf of Alaska (n=28), the Celtic-Biscay Shelf (n=24) and the Newfoundland-Labrador Shelf (n=21)

\subsubsection*{Taxonomy}

Assessments for REF:SQL:TOTNUMASSESSBYSPECIES species from
REF:SQL:TOTNUMASSESSBYFAMILY families and REF:SQL:TOTNUMASSESSBYORDERS
orders are included in the database (Figure~\ref{fig:taxo:srdb}). Five
taxonomic orders (REF:SQL:NUMASSESSORDERS:1,
REF:SQL:NUMASSESSORDERS:2, REF:SQL:NUMASSESSORDERS:3,
REF:SQL:NUMASSESSORDERS:4 and REF:SQL:NUMASSESSORDERS:5) account for
80\% of available stock assessments.  Of these, Perciformes, the most
speciose Order of marine fishes are in fact underrepresented in the
database (46\% of all marine fish species vs.  19\% of all marine
fish assessments), while the other four orders are
taxonomically overrepresented: Clupeiformes (2.1\% of marine fishes
vs.  11\% in the database), Gadiformes (3.3\% of marine fishes vs.
21\% in the database), Pleuronectiformes (4.5\% of marine fishes vs.
17\% in the database), Scorpaeniformes (8.5\% of marine fishes vs.
12\% in the database) (Figure~\ref{fig:taxo:threepanel}).

Assessed marine fish stocks in the RAM Legacy database constitute a
relatively small proportion of harvested taxa
(REF:SQL:SRDBPERCENTSPPSAUP\% of fish species from the SAUP database)
and an even smaller proportion of marine fish biodiversity
(REF:SQL:SRDBPERCENTSPPFISHBASE\% of fish species in FishBase;
Figure~\ref{fig:taxo:threepanel}). In turn, catches from the SAUP
database, which come from REF:SQL:SAUPNUMSPP species and
REF:SQL:SAUPNUMORDERS orders (Figure~\ref{fig:taxo:threepanel}),
represent only REF:SQL:SAUPPERCENTSPPFISHBASE\% of the
REF:SQL:NUMSPPFISHBASE species and REF:SQL:SAUPPERCENTORDERSFISHBASE\%
of the REF:SQL:NUMORDERSFISHBASE different orders present in FishBase
(Figure~\ref{fig:taxo:threepanel}). The diversity of harvested marine
invertebrates is clearly underrepresented in the stock assessment
database and likely in stock assessments in general.

%The paucity of marine invertebrate stock assessments means these
%species are more poorly taxonomically represented in the database than
%fishes (Figure 4XX). Only XX\% .....

%\subsubsection*{Global Fisheries}

%Table~\ref{tab:worldfisheries}

\subsubsection*{Ecology}
Assessed species span a range of ecological traits. Some life-history
information (e.g. growth, maturity, fecundity) is available for
REF:SQL:NUMASSESSLIFE of the collated assessments. In some cases, this
information is derived from biological studies, while in other cases
life-history parameters represent model assumptions (e.g., natural
mortality = 0.2) or model estimates. 

%The trophic level of assessed
%species ranged from REF:SQL:MINTL to REF:SQL:MAXTL with a mean of
%REF:SQL:MEANTL, with no apparent relationship between trophic level and stock status (Figure~\ref{fig:TL}).

%%Of these, age at sexual maturity ranged from XX to XX (n=XX, mean=XX) and . 
%was REF:SQL:NUMASSESSLIFE.

%Assessed species in the data span a range of ecological
%traits..... [need trophic level plot here; Figure 6). -assessemnts by
%trophic level: can we make a barplot showing number of stocks by
%trophic level for a) overexploited, and b) not overexploited species
%(i.e. could be on same plot but different hatching for the a vs. b.


\subsubsection*{Timespan }

%Of the REF:SQL:TOTNUMASSESSMENT stock assessments, time series data of
%catch/landings were available for REF:SQL:NUMCATCHSERIES stocks (REF:SQL:PERCENTCATCHSERIES\%),
%of SSB for REF:SQL:NUMSSBSERIES stocks (REF:SQL:PERCENTSSBSERIES\%), and of recruitment for
%REF:SQL:NUMRSERIES stocks (REF:SQL:PERCENTRSERIES\%).  

The median lengths of catch/landings, SSB, and recruitment timeseries
were REF:SQL:MEDCATCHLEN, REF:SQL:MEDSSBLEN, and REF:SQL:MEDRLEN
years, respectively (Figure~\ref{fig:orca}).  The time period covered by 90\% of assessments
is: catch/landings (REF:SQL:CATCHTIMESPAN90), SSB
(REF:SQL:SSBTIMESPAN90), recruitment (REF:SQL:RTIMESPAN90), while that
covered by 50\% of assessments is: catch/landings
(REF:SQL:CATCHTIMESPAN50), SSB (REF:SQL:SSBTIMESPAN50), recruitment
(REF:SQL:RTIMESPAN50) (Figure~\ref{fig:orca}).

\subsubsection*{Stock assessment methodologies and BRPs}
%In addition to the REF:SQL:TOTNUMASSESSMENT assessments in the
%database, indices of relative abundance from scientific surveys are
%available for an additional REF:SQL:TOTNUMWITHOUTMODEL stocks. 

The three most common assessment methods were
REF:SQL:NUMASSESSMETHOD:1, REF:SQL:NUMASSESSMETHOD:2 and
REF:SQL:NUMASSESSMETHOD:3. Regionally, Virtual Population Analysis
(VPA) is still the most common assessment model for European stocks
(REF:SQL:PERCENTVPAICES\% of REF:SQL:TOTNUMASSESSICES assessments),
Canada (REF:SQL:PERCENTVPADFO\% of REF:SQL:TOTNUMASSESSDFO
assessments) and Argentina (REF:SQL:PERCENTVPACFP\% of
REF:SQL:TOTNUMASSESSCFP assessments), whereas statistical catch-at-age
and -length models are more common for the United States
(REF:SQL:PERCENTSCALNMFS\% of REF:SQL:TOTNUMASSESSNMFS assessments),
Australia (REF:SQL:PERCENTSCALAFMA\% of REF:SQL:TOTNUMASSESSAFMA
assessments) and New Zealand (REF:SQL:PERCENTSCALMFISH\% of
REF:SQL:TOTNUMASSESSMFish assessments).
% (need to add a sentence here about the regional differences). 

Biomass- or exploitation-based reference points were available for
REF:SQL:NUMASSESSBIOREF (REF:SQL:PERCENTASSESSBIOREF\%) and
REF:SQL:NUMASSESSEXPLOITREF (REF:SQL:PERCENTASSESSEXPLOITREF\%)
assessments, respectively. The most commonly reported biomass-based
BRPs relate to biomass at MSY (e.g. $B_{msy}$), to ``limit'' biomass
(e.g. $B_{lim}$, a biomass level above which stocks should be
maintained) and to ``precautionary approach'' biomass (e.g.  $B_{pa}$,
a biomass level which provides an additional buffer to account for
uncertainty). Biomass and exploitation of United States' stocks under
the management of NMFS must follow MSY-based reference points whereas
other fisheries agencies use different BRPs.

\subsubsection*{Global Fisheries}
Assessments were available for 9 of the 10 largest fisheries for
individual fish stocks globally (Table~\ref{tab:worldfisheries}).
Assessments for Japanese anchovy in the East China Sea (the eighth largest species
for an individual stock, and tenth overall) were not publicly
accessible. Looking more broadly, the database contains assessments
for 17 of the 30 largest fisheries for individual fish stocks
globally, and 18 of the 40 largest fisheries globally (including those
recorded at lower taxonomic resolutions)
(Table~\ref{tab:worldfisheries}). Many of the fisheries not included
in the RAM Legacy database, especially those recorded in the SAUP
database as ``Marine fishes not identified'' (n=7), occur in
developing countries and have no known formal stock assessment
conducted for them.  From a national perspective, assessments are only
included for 2 of the top 10 wild-caught marine fisheries producing
nations, U.S.A. and Russia \citep{FAO:sofia}, with only two
assessments from the latter. We were unable to obtain any assessments
from the other top 10 yield-producing countries: China, Peru,
Indonesia, Japan, Chile, India, Thailand, Philippines
\citep{FAO:sofia}.


% cushion by using "estimates"

\subsubsection*{The status of commercially exploited marine stocks }
To evaluate stock status, we single out stocks for which both a biomass
BRP and an exploitation BRP are available. Of the
REF:SQL:NUMASSESSFRIEDEGG stocks presented in
Figure~\ref{fig:friedegg}, REF:SQL:NUMASSESSBIOASSESSREF and
REF:SQL:NUMASSESSBIOSCHAEFERREF of the biomass reference points and
REF:SQL:NUMASSESSEXPLOITASSESSREF and
REF:SQL:NUMASSESSEXPLOITSCHAEFERREF of the exploitation reference
points come from assessments and from surplus production model fits,
respectively.  To identify potential biases arising from using BRPs
derived from surplus production models we computed a contingency table
of status classification for stocks that have both assessment- and
Schaefer-derived BRPs (Table S2). Surplus production models correctly
classified ratios of current biomass to BRPs in
REF:SQL:PERCENTCORRECTBCLASS\% of cases (for REF:SQL:NUMCORRECTBCLASS
of REF:SQL:NUMBCLASS assessments) and REF:SQL:PERCENTCORRECTUCLASS\%
of cases for exploitation BRPs (for REF:SQL:NUMCORRECTUCLASS of
REF:SQL:NUMUCLASS assessments).

%For the stocks where both are available, we found a correlation of REF:SQL:PERCENTCORRBBMSY and REF:SQL:PERCENTCORRFFMSY between assessment BRP and surplus production model BRP for biomass (n=REF:SQL:NUMFORCORRBBMSY) and exploitation (n=REF:SQL:NUMFORCORRFFMSY), respectively (Supplementary Figure S1). 

% BRPs derived from surplus production models tended to underestimate $B/B_{msy}$ and overestimate $U/U_{msy}$.

Overall, REF:SQL:PERCENTASSESSMENTSBELOWBMSY\% of stocks are estimated
to be below their biomass-related MSY BRP, that is $B_{curr}<B_{msy}$,
and REF:SQL:PERCENTASSESSMENTSABOVEFMSY\% are estimated to be above
their exploitation-related MSY BRP, $U_{curr}>U_{msy}$
(n=REF:SQL:NUMASSESSFRIEDEGG stocks total; Figure~\ref{fig:friedegg}).
Of the stocks for which biomass is currently estimated to be below
$B_{msy}$, REF:SQL:PERCENTASSESSBELOWBMSYANDBELOWFMSY\% have had their
exploitation rate reduced below $U_{msy}$, suggesting potential for
recovery (Figure~\ref{fig:friedegg}). The remaining
REF:SQL:PERCENTASSESSBELOWBMSYANDABOVEFMSY\% of these stocks however,
still have excessive exploitation rates (Figure~\ref{fig:friedegg}).
On a positive note, REF:SQL:PERCENTASSESSMENTSABOVEBMSY\% of all stocks are
estimated to be above $B_{msy}$, and
REF:SQL:PERCENTASSESSABOVEBMSYANDBELOWFMSY\% of the stocks above
$B_{msy}$ also have $U_{current}$ below $U_{msy}$.

% put contingency table instead of the correlation

% surplus production model systematically provide 

%There was no significant
%difference in the status of stocks with assessment-derived BRPs (n=62,
%solid dots in Figures ~\ref{fig:friedegg} and ~\ref{fig:friedeggmgmt}) vs. Schaefer-estimated BRPS (n=178,
%open circles in Figures ~\ref{fig:friedegg} and ~\ref{fig:friedeggmgmt; p<XX).).

The status of exploited marine stocks, as estimated from biomass- and
exploitaion-BRPs, varied widely depending on the management body
(Figure~\ref{fig:friedeggmgmt}). Most European stocks (managed by
ICES) have biomasses less than $B_{msy}$
(REF:SQL:PERCENTASSESSFRIEDBELOWBMSYICES\%), and over half of these
stocks (REF:SQL:PERCENTASSESSFRIEDBELBMSYANDABOVEFMSYICES\%) still
have exploitation rates exceeding $U_{msy}$. Canadian stocks (managed
by DFO) also had low biomass (REF:SQL:PERCENTASSESSFRIEDBELOWBMSYDFO\%
$< B_{msy}$), but all but one of these has had its exploitation rate
reduced below $U_{msy}$. In contrast, about half
(REF:SQL:PERCENTASSESSFRIEDABOVEBMSYNMFS\%) of U.S. stocks (managed by
NMFS) are estimated to still be above $B_{msy}$, and of the
REF:SQL:NUMASSESSFRIEDBELOWBMSYNMFS stocks that are below $B_{msy}$
REF:SQL:PERCENTASSESSFRIEDBELBMSYANDBELOWFMSYNMFS\% have exploitation
rates below $U_{msy}$ (Figure~\ref{fig:friedeggmgmt}). In the New
Zealand and Australian waters, stocks managed by MFish and AFMA are
above $B_{msy}$ in REF:SQL:PERCENTASSESSFRIEDABOVEBMSYMFish\% and
REF:SQL:PERCENTASSESSFRIEDABOVEBMSYAFMA\% of cases, respectively. For
the stocks grouped as ``Atlantic'' in Figure~\ref{fig:friedeggmgmt} we
found that REF:SQL:NUMASSESSFRIEDBELOWBMSYICCAT of the
REF:SQL:NUMASSESSFRIEDTOTALICCAT ICCAT stocks and
REF:SQL:NUMASSESSFRIEDBELOWBMSYICCAT of the
REF:SQL:NUMASSESSFRIEDTOTALICCAT of NAFO stocks were below $B_{msy}$ .

%Species under international management include tuna stocks in the
%Atlantic, Pacific, Indian Oceans. 


%From these assessments,
%REF:SQL:NUMASSESSBIOANDEXPLOITREF report both a biomass-based and an
%exploitation-based BRP and appear as solid dots on
%Figures~\ref{fig:friedegg} and ~\ref{fig:friedeggmgmt}.
%Schaefer-derived BRPs add an additional
%REF:SQL:NUMADDITIONALASSESSSCHAEFER assessments, for a total of
%REF:SQL:NUMASSESSFRIEDEGG assessments used to generate
%Figure~\ref{fig:friedegg}. Overall,
%REF:SQL:PERCENTASSESSMENTSBELOWBMSY\% of assessed stocks are below
%their biomass-related MSY BRP and
%REF:SQL:PERCENTASSESSMENTSABOVEFMSY\% are above their
%exploitation-related MSY BRP. Different management bodies have
%different overall status of current biomass to BRPs
%(Figure~\ref{fig:friedeggmgmt}).


%Status of Assessed Stocks 
%Need to know:
%\% of stocks with biomass below Bmsy
%\% of stocks with
%overall and by management body.
 

