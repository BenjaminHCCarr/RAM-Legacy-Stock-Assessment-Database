


Marine wild capture fisheries provide more than 80 million tons of
fisheries products (both food and industrial) per year and employ 43.5
million people (wild capture and aquaculture, \citep{FAO:sofia}).
At the same time, fishing has been recognized as having one of the
most widespread human impacts in the world's oceans
\citep{Halpern:etal:2008:science}, and the Food and Agricultural
Organization of the United Nations (FAO) estimates that two-thirds of
fish stocks globally are fully exploited or overexploited
\citep{FAO:sofia}. While many fisheries have reduced exploitation
rates to levels that should in theory promote recovery, overfishing continues to
be a serious global problem \citep{Worm:etal:2009:science}. Fishery
managers are asked to address multiple competing objectives, including
maximizing yields, ensuring profitability, reducing bycatch, and
minimizing the risk of overfishing. Given the large social and
economic costs \citep{Rice:etal:2003:icescm} and ecosystem
consequences \citep{Frank:etal:2005:science, Myers:etal:2007:science}
of collapsed fisheries, it is imperative that we are able to quickly
learn from successful and failed fisheries from around the world.

Effective management of exploited fish populations generally requires
an understanding of where the current size and harvest rate lie in
relation to the size and rate which maximize fishery benefits or limit
the risk of overfishing. This process of quantitative determination of
stock status and estimation of reference points is called stock
assessment. Some fisheries in developing countries have apparently
provided sustainable yields for long periods of time without formal
stock assessment (e.g. many community-managed fisheries in Oceania;
\cite{Dalzell:1998:coastmgmt}). This sustainability has been achieved
by limiting harvest rates, often through gear restrictions or seasonal
and area closures. In modern industrialized fisheries, however, where
fishing capacity often exceeds the productivity of fished stocks,
stock assessment is an integral component of responsible management
\citep{Hilborn:Walters:1992}.

The global databases of fishery landings compiled by
\cite{FAO:fishstat} and extended by the Sea Around Us project
\citep{Watson:etal:2004:fandf} have proven to be valuable resources
for understanding the status of fisheries worldwide; however, catch
data alone can be misleading when used as a proxy for stock size.
Many papers have used these data to examine changes in fishery status
\citep{Worm:etal:2006:science, Costello:etal:2008:science}, including
changes in trophic level \citep{Pauly:etal:1998,
  Essington:etal:2006:procnatacadsci, Newton:etal:2007:currentbiol}.
Most of these analyses rely (either explicitly or implicitly) on the
assumption that catch or landings is a reliable index of stock size.
Critics have pointed out that catch can change for a number of reasons
unrelated to stock size, including changes in targeting, fishing
restrictions, or market preferences \citep{deMutsert:etal:2008:pnas,
 Murawski:Methot:Tromble:2007:science, Hilborn:2007:science,
 Caddy:etal:1998:science}. Standardizing catch by the amount of
fishing effort (catch-per-unit-of-effort, CPUE) is an improvement,
particularly when these data are modeled to account for spatial,
temporal, and operational factors affecting the CPUE
\citep{Maunder:punt:2004:fishres}, but CPUE can still be an unreliable
index of relative abundance since it is difficult to account for all
relevant factors \citep{Hutchings:Myers:1994:cjfas,
 Harley:etal:2001:cjfas, Walters:2003:cjfas, Polacheck:2006:marpol}.

Stock assessments consider time series of catch along with other
sources of information such as: natural mortality rates, changes in
size or age composition, stock-recruitment relationships, and CPUE
coming from different fisheries and/or from fishery-independent surveys.
Because they integrate across multiple sources of information, stock
assessment models should provide a more accurate picture of
changes in abundance than catch data alone
\citep{Sibert:etal:2006:science}, the trade-off being that their complexity
renders them difficult for non-experts to evaluate. Without a current and
comprehensive database of stock assessments, however scientists wishing to
conduct comparative analyses of marine fish population dynamics and
fishery status have little choice but to use problematic catch data.

