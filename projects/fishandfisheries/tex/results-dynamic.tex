\newpage
\section*{Results}
%\subsection*{Scope of the RAM Legacy database}
\subsection*{The knowledge-base for commercially-exploited marine stocks}
In total, 324 recent stock assessments for
288 marine fish and 36
invertebrate populations are included in the RAM Legacy database
(Version 1.0, 2010; Table S1). Together these comprise time series of
catch/landings were available for 308 stocks (95\%),
of SSB for 271 stocks (84\%), and of recruitment for
269 stocks (83\%) (Table S1).

\subsubsection*{Management bodies and geography}
Stock assessments are derived from fisheries management bodies
in Europe, the United States, Canada, New Zealand, Australia, Russia,
South Africa and Argentina (Table~\ref{tab:mgmt}). Also included are assessments
conducted by eight Regional Fisheries Management Organizations
(RFMOs), in the Northwest Atlantic, Atlantic, Pacific and Indian Ocean
(Table~\ref{tab:mgmt}). Assessments from the United States comprise by far the most
stocks of any country or region (n=139);
assessments from the European Union's management body ICES comprise
the the second greatest number of stocks (n=63).
Whereas nations are responsible for managing all populations within
their EEZs, RFMOs typically focus on a certain type of species (e.g.
halibut, tunas) or fisheries (e.g. pelagic high seas) within a given
area and hence assess a smaller number of stocks.

Most assessments come from North America, Europe, Australia, New
Zealand and the High Seas, while few are entered from regions such as
Southeast Asia, South America, and the Indian Ocean (outside
Australian waters) (Figure~\ref{fig:lmes}). Assessments were available for 31 LMEs, with the greatest number of
assessed stocks coming from Northeast U.S. Continental Shelf (n=58),
California Current (n=35), New Zealand Shelf (n=29),
Gulf of Alaska (n=26), Celtic-Biscay Shelf (n=26), East Bering Sea (n=22)
and Southeast U.S. Continental Shelf (n=20) (Figure~\ref{fig:lmes}).

%Northeast U.S. Continental Shelf (n=88), the California Current (n=35, the East Bering Sea (n=32), the New Zealand Shelf (n=29), the Gulf of Alaska (n=28), the Celtic-Biscay Shelf (n=24) and the Newfoundland-Labrador Shelf (n=21)

\subsubsection*{Taxonomy}

157 species from
57 families and 20
orders are included in the database (Figure~\ref{fig:taxo:srdb}). Five
taxonomic orders (Gadiformes (n=67),
Perciformes (n=62), Pleuronectiformes (n=53),
Scorpaeniformes (n=41) and Clupeiformes (n=36)) account for
80\% of available stock assessments.  Of these, Perciformes, the
most speciose Order of marine fishes are in fact underrepresented in
the database (46.04\% of all marine fish species vs.  19\%
of all marine fish assessments), while it is the other four orders
that are taxonomically overrepresented: Clupeiformes (2.1\% of marine
fishes vs.  11\% in the database), Gadiformes (3.3\% of marine fishes
vs.  21\% in the database), Pleuronectiformes (4.5\% of marine fishes
vs.  17\% in the database), Scorpaeniformes (8.5\% of marine fishes
vs. 12\% in the database) (Figure~\ref{fig:taxo:threepanel}).

Assessed marine fish stocks in the RAM Legacy database comprise a
relatively small proportion of harvested taxa (24\% of fish species
from the SAUP database) and an even smaller proportion of marine fish
biodiversity (1\% of fish species in FishBase;
Figure~\ref{fig:taxo:threepanel}). In turn, catches from the SAUP
database, which come from 649 species and
36 orders (Figure~\ref{fig:taxo:threepanel}),
represent only 5\% of the 12339 species and 67\% of
the 54 different orders present in FishBase
(Figure~\ref{fig:taxo:threepanel}).

%The paucity of marine invertebrate stock assessments means these
%species are more poorly taxonomically represented in the database than
%fishes (Figure 4XX). Only XX\% .....

%\subsubsection*{Global Fisheries}

%Table~\ref{tab:worldfisheries}

\subsubsection*{Timespan }

Of the 324 stock assessments, time series data of
catch/landings were available for 308 stocks (95\%),
of SSB for 271 stocks (84\%), and of recruitment for
269 stocks (83\%).  The median lengths of
catch/landings, SSB, and recruitment timeseries were
38, 34, and 33 years,
respectively.  The time period covered by 90\%
of assessments is: catch/landings (1967-2007), SSB (1972-2007), recruitment
(1971-2006), while that covered by 50\% of assessments is: catch/landings
(1983-2004), SSB (1985-2005), recruitment (1984-2003) (Figure~\ref{fig:orca}).

\subsubsection*{Stock assessment methodologies and BRPs}
In addition to the 324 assessments in the
database, indices of relative abundance from scientific surveys are
available for an additional 26 stocks. The
three most common assessment methods were Statistical catch-at-age/length models (n=164),
Virtual Population Analyses (n=90) and Biomass dynamics model (n=45)..  .....Need
to add a sentence here about the regional differences.... .


The total number and percentage of assessments that reported biomass-
or exploitation- based reference points of any sort was
257 (81\%) and
222 (69\%),
respectively. From these assessments,
66 report both a biomass-based and an
exploitation-based BRP and appear as solid dots on
Figures~\ref{fig:friedegg} and ~\ref{fig:friedeggmgmt}. Schaefer-derived BRPs
add an additional 175 assessments, for
a total of 241 assessments used to generate
Figure~\ref{fig:friedegg}. Overall,
58\% of assessed stocks are below
their biomass-related MSY BRP and
30\% are above their
exploitation-related MSY BRP. Different management bodies have
different overall status of current biomass to BRPs
(Figure~\ref{fig:friedeggmgmt}).
%Status of Assessed Stocks 
%Need to know:
%\% of stocks with biomass below Bmsy
%\% of stocks with
%overall and by management body.
 
\subsubsection*{Ecology}
Assessed species span a range of ecological traits. In terms of their
trophic level, we see XXX (Figure~\ref{fig:TL}). The total number and
percentage of assessments that reported any life-history information
(growth, maturity, fecundity) was 288.


%Assessed species in the data span a range of ecological
%traits..... [need trophic level plot here; Figure 6). -assessemnts by
%trophic level: can we make a barplot showing number of stocks by
%trophic level for a) overexploited, and b) not overexploited species
%(i.e. could be on same plot but different hatching for the a vs. b.


