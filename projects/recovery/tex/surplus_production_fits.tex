\documentclass[12pt]{proc}
\usepackage[top=.5in, nohead, nofoot]{geometry}
\usepackage[latin1]{inputenc}
%\usepackage[T1]{fontenc}
\usepackage[dvips]{graphicx}
\usepackage{appendix}
\usepackage{graphics}
\usepackage{amsmath, amsthm,natbib}
\usepackage{amssymb, latexsym}
\usepackage{textcomp}
\usepackage{verbatim}
\usepackage{setspace}
\usepackage{graphics}
\usepackage{wasysym}
\usepackage{float}
\usepackage{color}
\usepackage{rotating}
\usepackage{verbatim}
\usepackage{listings}
\setcounter{secnumdepth}{-1}
\author{C. Minto}
\title{Biomass dynamic model fit comparisons}
\date{}
\begin{document}
\maketitle
\citet{WaltersHilborn1976} presented a difference form of the continuous Schaefer
\begin{equation}
B_{t+1}=B_{t}+rB_{t}\left( 1-\frac{B_{t}}{k}\right)-qE_{t}B_{t}
\label{eqn:schaefer1}
\end{equation}
where $B_t$ is the biomass at time $t$, $r$ is the population growth rate, $k$ is the carrying capacity, $q$ is catchability, and $E_{t}$ is effort. From the assessment database, estimates of the biomass $(\hat{B_{t}})$ and catch $(\hat{C_{t}})$ are provided. For clarity, equation~(\ref{eqn:schaefer1}) is re-written  
\begin{equation}
 \hat{B}_{t+1}=\hat{B_{t}}+r\hat{B_{t}}\left( 1-\frac{\hat{B_{t}}}{k}\right)-\hat{C_{t}}
\label{eqn:schaefer2}
\end{equation}
Estimates of the uncertainty are not generally available so the $\hat{B_t}$ and $\hat{C_t}$ are treated as knowns. The question is, how to obtain parameter estimates and resulting reference points from these series? \citet{WaltersHilborn1991} suggest two methods: regression and timeseries fitting.
\subsection{Regression}
Surplus production is given by
\begin{equation}
 S_{t}=B_{t+1}-B_{t}+C_{t}
\end{equation}
where $S_{t}$ is the surplus production at time $t$ \citep{Hilborn2001}. Re-formulating equation~(\ref{eqn:schaefer2})
\begin{equation}
\hat{S}_{t}=\hat{B}_{t+1}-\hat{B}_{t}+\hat{C}_{t}=rB_{t}\left( 1-\frac{B_{t}}{k}\right)
\end{equation}
which can be written as a surplus production rate over biomass.
\begin{eqnarray}
\frac{\hat{S}_{t}}{B_{t}}&=&r\left( 1-\frac{B_{t}}{k}\right)\\
&=&r-\frac{r}{k} B_{t}
\end{eqnarray}
\bibliographystyle{./coilin.bst}
\bibliography{/home/srdbadmin/SQLpg/srDB/doc/srdb.bib}
\end{document}
 
The surplus production formula presented in 
