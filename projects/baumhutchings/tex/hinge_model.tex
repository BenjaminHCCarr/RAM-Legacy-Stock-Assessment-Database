\documentclass[a4paper]{article}
\usepackage[latin1]{inputenc}
\usepackage{amsmath}
\usepackage{amsfonts}
\usepackage{amssymb}
\usepackage{longtable}
\usepackage{booktabs}
\usepackage[top=2cm, bottom=2cm, right=2cm, left=2cm]{geometry}
\usepackage{lscape}
\usepackage{times}
\title{Hinge model}
\date{}
\begin{document}
\maketitle
\noindent To investigate the hinge model, we'll look at the deterministic part, given by
\begin{eqnarray}
\ln(SSB_{t})=\begin{cases}
\alpha_1+\beta_1 t & \text{if $t \le C$ },\\
\alpha_2+\beta_2 t& \text{if $t > C$}.
\end{cases}
\label{eqn:hinge1}
\end{eqnarray}
For $\ln(SSB_{C})$ to be continuous requires that 
\begin{eqnarray}
\alpha_1+\beta_1 C &= &\alpha_2+\beta_2 C\\
\alpha_2&=&\alpha_1+\beta_1 C-\beta_2 C\\
&=&\alpha_1+ C(\beta_1-\beta_2 )
\end{eqnarray}
Substituting back into Equation~(\ref{eqn:hinge1})
\begin{eqnarray}
\ln(SSB_{t})=\begin{cases}
\alpha_1+\beta_1 t & \text{if $t \le C$ },\\
\alpha_1+ C(\beta_1-\beta_2 )+\beta_2 t &\text{if $t > C$}.
\end{cases}
\label{eqn:hinge2}
\end{eqnarray}
Define
\begin{eqnarray}
\beta_2&=&\beta_1+\delta_{\beta}\\
\delta_{\beta}&=&\beta_2-\beta_1
\end{eqnarray}
Substitute in Equation~(\ref{eqn:hinge2})
\begin{eqnarray}
\ln(SSB_{t})&=&\begin{cases}
\alpha_1+\beta_1 t & \text{if $t \le C$ },\\
\alpha_1+ C(-\delta_{\beta})+(\beta_1+\delta_{\beta}) t &\text{if $t > C$}.
\end{cases}\\
&=&\begin{cases}
\alpha_1+\beta_1 t & \text{if $t \le C$ },\\
\alpha_1 +\beta_1 t +\delta_{\beta}(t-C)&\text{if $t > C$}.
\end{cases}
\label{eqn:hinge3}
\end{eqnarray}
Define
\begin{eqnarray}
\eta_{t}=\begin{cases}
0 & \text{if $t \le C$ },\\
1 &\text{if $t > C$}.
\end{cases}
\end{eqnarray}
\begin{equation}
\ln(SSB_{t})=\alpha_1+\beta_1 t + \eta_{t}\delta_{\beta}(t-C).
\end{equation}
\end{document}






