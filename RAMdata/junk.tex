U. Sk�lad�ttir, 2001.
The Icelandic Shrimp Fishery (Pandalus borealis Kr.) at Flemish Cap in 1993-2001

Northwest Atlantic Fisheries Organization
 NAFO SCR Doc. 01/183

NOT TO BE CITED WITHOUT PRIOR
REFERENCE TO THE AUTHOR(S)
Northwest Atlantic Fisheries Organization
Serial No. N4573 NAFO SCR Doc. 01/183
SCIENTIFIC COUNCIL MEETING � NOVEMBER 2001
The Icelandic Shrimp Fishery (Pandalus borealis Kr.) at Flemish Cap in 1993-2001
by
Marine Research Institute, Sk�lagata 4,
P.O. Box 1390, 121 Reykjav�k, Iceland
Abstract
Some 4 Icelandic vessels have been fishing for shrimp in the waters at Flemish Cap in 2001 compared to 7 in
2000. In this paper there are logbook information on the Icelandic fishery for the years 1993 through 2001. The
standardized catch rate has recently increased considerably or from 192 kg/hour in January-July 1997 to 294 in 1998
and was 252 and 245 kg in 1999 and 2000 respectively to rise to 295 kg/hour, the second highest since 1993.
The observer samples show a very strong year-class of 2 year olds appearing in September 2001.
Introduction
The Spanish investigators (EU) have been measuring the biomass index of northern shrimp at the Flemish
Cap since 1988 in their annual bottom trawl survey at Flemish Cap. In 1993 the fishery was initiated by Canada,
followed closely by Faroe Islands and Iceland.
The fishery was some 24-33 thousand tons in the years 1993-1995 to increase in 1996 to 48 thousand tons. Since then
the fishery decreased to some 25 thousand tons in 1997. The total catch of all countries has since increased to about
50 thousand tons in 2000.
In this paper all the information from the Icelandic side is gathered. From the logbooks comes effort, catch
and size of trawl. From this CPUE is calculated. From the biological samples taken by Icelandic observers comes
various information on length and sex distribution of shrimp. From these the age assessments can be carried out.
There is also detailed information on length frequency distributions by depth strata.
Materials and Methods
The logbook data include catch and effort. Not all skippers send in the logbooks, but information on
landings can be obtained from the Fisheries Directorate in Iceland. Thus effort was raised by dividing the nominal
catch of each month with the calculated CPUE from the logbooks in the years 1993-1996. In 1997 and the effort is first
raised to the nominal effort by every half year. The overall CPUE of the January-July was then obtained by summing
nominal catch of all months and corresponding effort. Nominal catch for the whole period was then divided by
"nominal effort" to get the CPUE for the period January-July. When twin trawls were used the effort was always
multiplied by 1.9 for those but the catch was kept the same.
Icelandic observers sampled shrimp onboard all Icelandic vessels in the years 1996 through 2000 at Flemish
Cap. The shrimp was measured fresh to the nearest 0.5 mm using Vernier calipers. Observers then sorted each length
class into males and females using the method of Rasmussen (1953) and the females further into primiparous and
multiparous using the sternal spine criterion of McCrary (1971).

