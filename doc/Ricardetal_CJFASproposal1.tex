\documentclass[10pt,oneside]{refletter}

\usepackage{url}
\usepackage{natbib}
\usepackage{longtable}
\usepackage{graphicx}
\usepackage[latin1]{inputenc}
\usepackage{makeidx} 
\pagestyle{plain}
\makeindex
\usepackage[T1]{fontenc}
%\usepackage[dvips]{graphicx}

\date{2008-11-14}

\begin{document}
% Proposal to CFJAS for a Rapid Communications articlae presenting the RAMII stock-recruitment database, with an extended online Supplementary Material where the plots (1 page per stock) can be presented 
%Include description of the database, appropriateness of publishing in CFAS, description of the paper's purpose and content
Department of Ecology and Evolutionary Biology,
University of Toronto,
Toronto, Ontario M5S 3G5, Canada
Email: don.jackson@utoronto.ca

November 17, 2008

Dear Dr. Jackson, 

We are writing to enquire about submitting an article for publication as a Rapid Communication in CJFAS based on a new global database of spawner and recruit data for marine fishes. 
	
The RAM II Stock Recruit database is a newly developed database inspired by the original Myers Stock Recruitment database built by Ransom Myers, Nick Barrowman, and Jessica Bridson in the mid-1990s (see http://chase.mathstat.dal.ca/~myers/welcome.html; \citep{Myersetal1995a}). That database was used in a large number of publications, including XX in the Canadian Journal of Fisheries and Aquatic Sciences, which lead to important advances in fisheries science and ecology. Increasingly, however, analyses are becoming limited because most of the time series data have not been updated in the past decade, and hence provide a potentially outdated picture of the status of marine fisheries.

The RAM II Stock Recruit database aims to meet the needs of fisheries scientists and marine ecologists interested in conducting broad-scale analyses of marine fish populations, by providing spawner biomass, recruitment, catch, reference point, and life history data from fisheries stocks assessments for all assessed marine fish populations in the world. We are developing the database in consultation with leading fisheries scientists and ecologists from Canada (Jeffrey A. Hutchings, Boris Worm), the U.S. (Ray Hilborn, Jeremy Collie, Mike Fogarty, Olaf Jensen), New Zealand (Pamela Mace), and Australia (Beth Fulton), as part of the National Centre for Ecological Analysis and Synthesis (NCEAS) working group ``Finding Common Ground in Marine Conservation and Management''. The database is implemented in a modern relational database management system using the Structured Query Language (SQL). To date, RAM II includes over 100 stock assessments and it is growing steadily (see \url{http://www.marinebiodiversity.ca/RAMlegacy/srdb} for additional details). 

In the proposed paper we will introduce the database to the scientific community, describe its purpose, document its structure, and briefly summarize the contents to date. As part of the content summary we would like to include a a single Supplementary Material file with a one-page summary for each stock showing the biomass and fishing mortality reference points, and plots of (i) temporal trend in spawning stock biomass and/or CPUE and fishing mortality, (ii) the stock-recruitment relationship with fitted Ricker and Beverton-Holt models, assuming gamma error structure (iii) a spawning stock biomass vs. fishing mortality phase diagram, and (iv) the temporal exploitation history of the stock using total catch and/or landings. 

We believe that it would be particularly appropriate to publish an introduction to this database in CJFAS because of the history of CJFAS publications based on the original Myers database. We also anticipate that there will be great interest by CJFAS readers in this database, both because of its relevance to their research and the growing interest in broad-scale empirical analyses. Indeed, we are already receiving enquiries from fisheries scientists and ecologists about the database, and we therefore believe it is important to publish an official introduction and documentation of the database which future users can cite. We expect that we could have a manuscript ready for submission by February 2009. 

Thank you for your consideration of this proposal. We look forward to hearing back from you at your earliest convenience. 

Sincerely,
Daniel Ricard, Coil\'in Minto and Julia K. Baum
\bibliography{Ricardetal_CJFASproposal}
\bibliographystyle{plain}
\end{document}
